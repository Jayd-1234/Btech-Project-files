\documentclass[a4paper,17pt]{extreport}
\usepackage[utf8]{inputenc}


\begin{document}


\begin{thebibliography}{9}
\bibitem{1}M. H. J. Bollen, Understanding Power Quality Problems: Voltage Sags
and Interruptions. New York: IEEE Press, 2000, Series on Power Engineering.
\bibitem{2} International Electrotechnical Vocabulary (IEV), International Electrotechnical
Commission Std. IEC 60 050, 1998.
\bibitem{3} Electromagnetic Compatibility (EMC), Part 4: Testing and Measurement
Techniques, Section 30: Power Quality Measurement Techniques,
International Electrotechnical Commission Std. IEC 61 000-4-30, 2000.
\bibitem{4}IEEE Recommended Practice for Monitoring Electric Power Quality,
IEEE Std. 1159, 1995.
\bibitem{5}IEEE Recommended Practice for Powering and Grounding Sensitive
Electronic Equipment, (Emerald Book), IEEE Std. 1100, 1992.
\bibitem{6}Electricity Supply — Quality of Supply, Part 1: Overview of Implementation
of Standards and Procedures, Part 2: Minimum Standards, Std.
NRS 048-1, -2, 1996.
\bibitem{7}Voltage Characteristics of the Electricity Supplied by Public Distribution
Systems, British/European Std. BS EN 50160, CLC, BTTF 68-6,
1994.
\bibitem{8}Electromagnetic Compatibility (EMC), Part 2: Environment, Section 8:
Voltage Dips and Short Interruptions on Public Electric Power Supply
Systems With Statistical Measurement, International Electrotechnical
Commission Std. IEC 61 000-2-8, 2000.
\bibitem{9}Specification for Semiconductor Processing Equipment Voltage Sag
Immunity, Semiconductor Equipment and Materials International Std.
SEMI F47-0200, 1999/2000.
\bibitem{10}Test Method for Semiconductor Processing Equipment Voltage Sag
Immunity, Semiconductor Equipment and Materials International: Std.
SEMI F42-0600, 2000.
\bibitem{11}Understanding SEMI F47 Voltage Sag Standard, SEMI F47 Std. Application
Note, 1999.
\bibitem{12}Electromagnetic Compatibility (EMC), Part 4: Testing and Measurement
Techniques, Section 11: Voltage Dips, Short Interruptions
and Voltage Variations Immunity Tests, British/European Std. BS EN
61000-4-11, 1994.
\bibitem{13}“Performance of a Hold-in Device for Relays, Contactors and Motor
Starters,” EPRI, Brief no. 46, 1998.
\bibitem{14}E. Collins and M. Bridgwood, “The impact of power system disturbances
on AC-coil contactors,” in Proc. IEEE Textile, Fiber Film Ind.
Tech. Annu. Conf., Greenville, SC, May 6–8, 1997.
\bibitem{15}“SEMATECH “Guide for the Design of Semiconductor Equipment to
Meet Voltage Sag Immunity Standards”,” Int. SEMATECH, Technology
Transfer Tech. Rep. #99 063 760B-TR, 1999.
\bibitem{16}B. Prokuda and A. Hernandez, “Improving process reliability due to
electrical interruptions Part I and Part II,” in Proc. Power Syst. World
Conf., Santa Clara, CA, Nov. 1998.
\bibitem{17}Schneider Electric: “Power Quality and the Application of the SEMI
Standard”. SquareD-semi. [Online]. Available: http://www.squaredsemi.
com/semif47page.asp#Zerocross
\bibitem{18}“DRTS-3 Waveform Generator,” Technical Documentation, Hathaway,
2002.
\bibitem{19}“Reliable Power Recorder, Model 1656,” Technical Documentation ,
Rhopoint Systems Ltd., 2002.
\bibitem{20}Monitoring Electric Power Quality, Task Force 3: Data File Format for
Power Quality Data Interchange, IEEE Std. 1150, Example PQDIF file,
2002.
\bibitem{21}(2002) “TOP, The Output Processor Software,” Samples. Electrotek
Concepts, PQ Soft. [Online]. Available: http://www.pqsoft.
com/top/download/TOPSamples.exe
\bibitem{22}R. M. Gnativ and J. V. Milanovic. Identification of voltage sag characteristics
from the measured response. presented at Proc. (CD Rom) 10th
Int. Conf. Harmonics Quality Power. [Online]

\end{thebibliography}

\end{document}
