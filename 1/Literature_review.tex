\documentclass[12pt, a4paper]{extreport}

\begin{document}
\begin{enumerate}
\item 
    \begin{enumerate}
        \item{\textbf{Name of the chapter:}} Impersonal probability assessment of equipment trip due to voltage sag. 

       \item{\textbf{Name of the author:}}Ying Wang, Yong Huang,Chao, Xianyong Xiao,college of electrical engineering. \linebreak
        
         \item{\textbf{Journal:}}2010, IEEE
        \item{\textbf{Summary:}}The paper has provided
an objective method,
using maximum entropy
principle to determine the
stochastic distribution of
any random event in
mathematical field, which
is a good alternative to
solve the probability
model of voltage tolerate
curve for given sensitive
equipment. As a case
study PC is simulated
under different conditions
and compared with the
traditional methods by
Monte Carlo.
\end{enumerate}
        
        \item 
        
        \begin{enumerate}
            
        \item{\textbf{Name of the chapter:}}A voltage sag
index
considering
compatibility
between
equipment and
supply
\item{\textbf{Name of the author:}}Cheng- Chieh
Shen, student
member, IEEE
Chan-Nan Lu,
senior member,
IEEE
\item{\textbf{Journal:}}IEEE
transactions
on power
delivery, Vol
22, No.2,
April 2007
\item{\textbf{Summary:}}This paper presents a
voltage sag index which
would be very useful for
indicating system
performance experience at
different locations. Fuzzy
logic techniques are
applied to quantify
voltage sag disturbance
level where retained
magnitude and sag
duration are the inputs to
the proposed system and
the output is an index
that indicated relative
severity of the
\end{enumerate}
\item \begin{enumerate}
\item{\textbf{Name of the chapter:}}Assessment of
financial losses
due to voltage
sag in an Indian
distribution
system
\item{\textbf{Name of the author:}}A.K Goswami,
student
member, IEEE
C P Gupta,
member, IEEE
G K Singh,
Department of
electrical
engineering,
IIT Roorkee
\item{\textbf{Journal:}}2008 IEEE
Region 10
colloquium
and the third
international
conference on
Industrial and
information
systems,
Kharagpur,
India,
December
2008
\item{\textbf{Summary:}}This paper presents a
practical implementation
of the stochastic
assessment of annual
financial losses due to
voltage sags considering
the uncertainties involved
with sensitivity and
interconnection of
equipments in individual
process, customer types
and location of the
process in system
network. The evaluation
of the impact of the
voltage sags at a
particular site in a
network involves two
basic steps – voltage sag
assessment and economic
assessment. The fault
positons method is used
for the stochastic
prediction of voltage sag
and for economic
assessment of customer
losses due to voltage sags,
it is assumed that every
nuisance trip of an
industrial process requires
24 hours of restoration
time. (Any other duration
adopted would not affect
the methodology used)The damage costs
reported by various
categories of customers
for 24 hours long
interruption are taken as
the damage costs for
process trips due to
voltage sags.

    \end{enumerate}
    
    \item
    \begin{enumerate}
    
            
        \item{\textbf{Name of the chapter:}Analysis of
Voltage sag
severity case study in an
industrial circuit
\item{\textbf{Name of the author:}}Santiago AriasGuzman,
student member, IEEE
Oscar AndresGuzman,
Maria
Jaramillo
Gonzales,
Pablo-Daniel
CardonaOrozco,
student
member, IEEE
Eduardo A.
Cano Plat,
senior member,
IEEE
Andres Felipe
Salazar-Jimenez
\item{\textbf{Journal:}}IEEE
transactions
on Industry application,
Vol 53, No.2,
January 2017.
\item{\textbf{Summary:}}This paper presents the
power study carried out in
an industrial distribution system. The main aim of
this study was to quantify
the negative impact
caused by voltage sags in
industrial process and its
relationship to generate
interruptions. Finally some
good practices of
industrial processes were
recommended. The
methodology used to
calculate the sag severity
is as follows: 1)Obtain
the records of waveforms
for a period of
measurement. 2)
Calculate the retained
voltage and duration of
voltage sags. 3)
Calculate the voltage sag
severity of voltage sags.
4) Calculate the voltage
sag severity on the studied
substation as the sum of
individual contributions of voltage sags.
\end{enumerate}
\item
    \begin{enumerate}
    \item{\textbf{Name of the chapter:}}Sensitivity of AC
coil contactors
to voltage sags,
short
interruptions
and
undervoltage
transients
\item{\textbf{Name of the author:}}Sasa Z. Djokic
Jovica V.
Milanovic´,
Senior Member,
IEEE, and
Daniel S.
Kirschen,
Senior Member,
IEEE
\item{\textbf{Journal:}}IEEE
transactions
on power
delivery,
vol.19, No.3,
July 2004
\item{\textbf{Summary:}}This paper discusses the
sensitivity of ac coil
contactors to voltage
sags, short interruptions
and undervoltage
transients on the basis of
the results of the
following tests: sensitivity
to rectangular voltage
sags with ideal and nonideal
supply voltage,
testing with two stage
voltage sags, voltage sags
which occur during
starting of large motors,
and also against the
measured voltage sag. All
contactors are tested
without load attached to
their main electrical
contacts. These normally
open main contacts are
used to get a clear
identification of the
disengagement of the
contactor. In testing, the
nominal voltage is applied
to the ac coil of the
contactor before and after
the voltage sag. If the sag
causes disengagement of
the contactor it surely
indicates a malfunction of
contactor.
\end{enumerate}
\item
    \begin{enumerate}
     \item{\textbf{Name of the chapter:}}Sensitivity of
personal
computers to
voltage sags and
short
interruptions
\item{\textbf{Name of the author:}}S. Z. Djokic,
J.Desmet,
member, IEEE
G. Vanalme,
J.V.Milanovic,
senior member,
IEEE, and
K.Stockman,
student
member, IEEE
\item{\textbf{Journal:}}IEEE
transactions
on power
delivery,
vol.20, No.1,
January 2005
\item{\textbf{Summary:}}This paper discusses the
sensitivity of personal
computers (PCs) to
voltage sags, short
interruptions on the basis
of the results of the
following tests: sensitivity
to rectangular voltage
sags with ideal and nonideal
supply voltage,
testing with two stage
voltage sags, voltage sags
which occur during
starting of large motors.
During the testing, a
rated voltage (having ideal
or non-ideal supply
characteristics) was
applied to the PCs before
and after the voltage sag.
The following
malfunctions are identified
(i) corruption or
interruption of read/write
drive. (ii) Frozen screen
and lack of response to
any command. (iii)
hardware malfunction.
    \end{enumerate}
    \item
    \begin{enumerate}
    \item{\textbf{Name of the chapter:}}Sensitivity of
Adjustable speed
drives (ASDs) to
voltage sags and
short
interruptions
\item{\textbf{Name of the author:}}S. Z. Djokic,
K. Stockman,
member, IEEE,
J.V.Milanovic,
senior member,
IEEE,
J. J. M.
Desmet,
member, IEEE
and
R. Belmans,
senior member,
IEEE
\item{\textbf{Journal:}}IEEE
transactions
on power
delivery,
vol.20, No.1,
January 2005
\item{\textbf{Summary:}}This paper discusses the
sensitivity of personal
computers (PCs) to
voltage sags, short
interruptions on the basis
of the results of the
following tests: sensitivity
to rectangular three
phase, two phase and
single-phase voltage sags
with ideal and non-ideal
supply characteristics, as
well as sensitivity to nonrectangular
balanced three
phase voltage sags similar
to those caused by
starting of large motors.

    
    \end{enumerate}
    \item
    \begin{enumerate}
    \item{\textbf{Name of the chapter:}}Probabilistic
assessment of
equipment trips
due to voltage
sags
\item{\textbf{Name of the author:}}C.P.Gupta,
member, IEEE,
J.V.Milanovic,
senior member,
IEEE
\item{\textbf{Journal:}}IEEE
transactions
on power
delivery,
vol.21, No.2,
April 2006
\item{\textbf{Summary:}}This paper discusses the
uncertainty involved in the
behavior of sensitive
equipment used in various
industrial processes and
methodology to
incorporate this effect in
quantifying the equipment
trips due to voltage sags
over a specified time
period. For the stochastic
assessment of equipment
trips due to voltage sags
the following four
different probable
behaviors were considered
within their ranges:-
(i) Uniform Sensitivity- If
there is an equal
probability that the
equipment voltage
tolerance curve may
assume any location
within the region of
uncertainty, it can be represented by assuming
fx(T) and fy(V) to be
uniform probability
density functions for Tand
V within their respective
range. (ii) Moderate
Sensitivity- This type of
sensitivity can be
expressed by assuming
fx(T) and fy(V) to be
normal probability density
functions. (iii) High
Sensitivity- If probabilities
are assumed in
exponentially decreasing
order from high-voltage
threshold to low-voltage
threshold and from low
duration threshold to high
duration threshold. (iv)
Low SensitivityExponential
distribution
assumed to be opposite of
the previous case. After
calculating the joint
probability density
functions, expected
number of trips of a
particular equipment can be calculated
    \end{enumerate}
    \item
    \begin{enumerate}
     \item{\textbf{Name of the chapter:}}Probabilistic
assessment of
financial losses
due to
interruptions
and voltage
sags-part I: The
methodology
\item{\textbf{Name of the author:}}J.V.Milanovic,
senior member,
IEEE
C.P.Gupta,
member, IEEE
\item{\textbf{Journal:}}IEEE
transactions
on power
delivery,
vol.21, No.2,
April 2006

    \item{\textbf{Summary:}}This paper provides a
generalized methodology
for the stochastic
assessment of the financial
losses due to interruptions
and voltage sags. The
methodology proposed
takes into account all the
uncertainties in a
probabilistic manner
associated with the
voltage sag calculation,
sensitivity of customers’
equipment to voltage
sags, the interconnection
of the equipment within
an industrial process, and
customer types and the
location of the process in
the network. For an
economic assessment of
financial losses due to
voltages sags, it is a
prerequisite to have the
information about the
type of
industrial/commercial
process, customer type,
and the associated
damage cost per sag event. Some of the
customers quote very high
cost for the single trip,
whereas for others, it
might not be that
substantial. Finally, the
total costs incurred due to
voltage sags and
interruptions
should be added together
in order to come up with
total network financial
losses for a given network topology
    \end{enumerate}
    \item
    \begin{enumerate}
    \item{\textbf{Name of the chapter:}}Probabilistic
assessment of
financial losses
due to
interruptions
and voltage
sags-part II:
practical
implementation
\item{\textbf{Nameof the author:}}J.V.Milanovic,
senior member,
IEEE
C.P.Gupta,
member, IEEE
\item{\textbf{Journal:}}IEEE
transactions
on power
delivery,
vol.21, No.2,
April 2006
\item{\textbf{Summary:}}The second part of this
paper presents a practical
implementation and
application of the
methodology for the
stochastic assessment of
the annual financial losses
due to interruptions and
voltage sags discussed in
the first part of this
paper. The methodology
is illustrated on a generic
realistic distribution
network with all relevant
network components
modeled appropriately.
Finally, different network
topologies are compared
taking into account total
financial losses in the network.
\end{enumerate}

     \item
    \begin{enumerate}
    \item{\textbf{Name of the chapter:}}Analysis of
voltage
tolerance of AC
adjustable-speed
drives for three
phase balanced
and unbalanced
sags
\item{\textbf{Name of the author:}}Math
H.J.Bollen,
senior member,
IEEE,
Lidong D.
Zhang
\item{\textbf{Journal:}}IEEE
transactions
on industry
applications,
Vol.36, No.3,
May/June
2000
\item{\textbf{Summary:}}This paper analyses the
behavior of three phase ac
adjustable speed drives,
equipment most sensitive
to voltage sags, during
balanced and unbalanced
sags. The conclusion from
the analysis is that voltage
sags due to three-phase
faults are a serious
problem for adjustablespeed
drives. However,
single-phase and phase-tophase
faults, causing the
majority of voltage sags,
can be tolerated by adding
a relatively small amount
of dc-bus capacitance. It
was shown that the drive
behavior during an
unbalanced sag is
completely different from
the behavior during a
balanced sag. It was also shown that the sag type
and the characteristic
magnitude are the main
affecting factors for the
drive behavior, with the
PN factor being important
in a limited number of 
cases.

    \end{enumerate}
     \item
    \begin{enumerate}
    \item{\textbf{Name of the chapter:}}Severity indices
for assessment
of equipment
sensitivity to
voltage sags and
short
interruptions

 \item{\textbf{Name of the author:}}J.Y. Chan,
Student
Member, IEEE,
J. V. Milanović,
Senior Member,
IEEE
\item{\textbf{Journal:}}IEEE 2007
\item{\textbf{Summary:}}The paper proposes new
indices to assess the
impact of voltage sags
and short interruptions on
industrial equipment. The
indices translate physical
sag parameters into terms
that reflect the severity of
the disturbances as seen
by the equipment. Using
these indices, the
uncertainties involved in
equipment tolerance are
addressed through proper
event categorization and a
common platform
developed for the
evaluation of equipment
ride through capability
with respect to voltage
sags and short
interruptions. The paper
introduces Magnitude
Severity Index (MSI),
Duration Severity Index
(DSI) and combined
Magnitude Duration
Severity Index (MDSI)
and Categorizes voltage
sags based on these
indices. Finally, it also
illustrates procedures to
convert physical sag
parameters (sag
magnitude and duration)
into severity indices on
programmable logic controllers (PLC),
personal computers (PC),
adjustable speed drives
(ASD) and AC contactors. 
    \end{enumerate}
         \end{enumerate}
\end{document}
